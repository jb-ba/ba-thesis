\section{Introduction - Frame it as a video IoT problem. IoT is nice but video in IoT is hard to manage especially live data. How to cope with a lot of real time data examplified in the leading video surveilance company.}
We live in an inter-connected world and nothing shows it better than the current explosion in IoT (source and numbers). This is only set to accelerate with the adoption of 5G (source). Many of these new devices are sensors continuously streaming data to the Internet. In many cases, this data is metadata and thus rather small. Streaming and consuming this data does not represent a challenge for powerful cloud servers. However, managing the cloud infrastructure is harder. This is, the programs responsible for handling the data. One new set of data can trigger multiple effects. In a distributed  in  . In contrast video cameras produce a large set of data and having  (maybe put latter stuff in motivation and concentrate on how much video is actually produced)


When uploading a file to a remote file storage like dropbox or google drive, the user relies on their services to guarantee storage and readability of these files. Even though from the users perspective these services offer a similar file structure to what the user has on his home computer, they are operate very differently. It is not 

Example youtube
a video on YouTube we rely on its service to not only store the video, but also to process it, scan it for copyright material and much more. As end users we do not care about the physical location of the data or the services, we just know YouTube takes care of everything. A similar transformation is happening to business and it is called the "cloud revolution". 



Put together, these expert views reveal the many frontiers on which cloud computing is driving
change: from internal operations to the IT industry, from the economy to the environment. That
diversity of change compels business and technology leaders not to think of cloud computing
simply as a replacement for older computing platforms. It is a revolution in the way information
is stored and shared that could prove as disruptive to business practices as the advent of
computing itself. 

\subsection{Motivation - More reference; Less technical --> building better scalability and flexbility, modular vs monolithic, cloud --> Don't be so technical, more highlevel and grandual go into real challenges}
Managing a large scale distributed system is still something the industry has to figure out how to do correctly. The advent of the microservice era, building many small applications each only concerned with its own logic, mitigated many problems of a monolithic approach, but also brought new challenges to the software architects. Docker and Kubernetes solved some of these, but mainly concerning the OS level abstraction and the configuration at build time. But only small and separated steps have been taken to manage the actual runtime of the microservices. This is what the service mesh is for. It is one of the hot topics in the industry right now and there are different solutions with similar approaches.
\\
The research question

\subsection{Problem Definition}