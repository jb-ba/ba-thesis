\clearpage
\section{Conclusion and Future Outlook}
As a system administator it is hard to describe what it means to have a system that works for you. Traditionally, deploying or updating an application meant locally updating the application on each edge device. A tedious and time consuming task.

Having a clearly defined APIs to tell the system what to do is a blessing. With Kubernetes deploying and updating applications can be done centrally via the configuration of API resources. The administrator tells the system what he wants and the system works for him. In the case of failure Kubernetes actively tries to recover from the failure and notifies the administrator that something is wrong. These and more are the advantages Kubernetes offers and the reason it is so popular in the cloud.

But it is not without its flaws. As Kelsey Hightower, a Kubernetes developer advocate at Google, said during KubeCon 2017 ``If you’re new to Kubernetes, it will set you free. But first it will piss you off''\cite{KubernetesPissYouOff:online}. From the learnings of this thesis, this is more than true. Understanding Kubernetes, let alone containers, can be challenging. Understanding and implementing industry best practices, like scratch containers, using the image digest and Helm for resource deployments can be daunting tasks. But once, the barrier of understanding and using Kubernetes and its tools is broken, Kubernetes opens up a new world for developers to manage resources.

Updating the application code for the implementation to add new features and security on remote devices like the RPi or any other node connected to the cluster is as easy as writing a yaml file. Additionally, rules can be defined to tightly control the system of how much resources components can use, who has access and the networking can be controlled independent of the application. In this project alone, which is quite simple, Kubernetes is already a blessing.  


If edge devices, especially based on ARM, can be integrated as first class citizen in the Kubernetes developer community, it has a huge potential to become the de-facto standard for edge computing.\\[5mm]

{\hspace*{5mm}{\textbf{\textit{Future Outlook}}}}

A few issues have to be worked out before Kubernetes should be used for production services on the edge. Istio needs better support for ARM. It is a crucial tool for fog computing as it allows automatic encryption between the nodes inside a cluster. A Kubernetes wide standard has to be implemented to deal with connection losses, retrials and longer timeouts to enable edge devices to function without a stable connection. One such solution is multi-cluster, but this feature is still in alpha. The IoT community needs to come together and try to reduce the number of IoT protocols so that specific protocols can be tightly integrated into Kubernetes.  

The Kubernetes IoT Edge working group\cite{IntroducingDejanBosanac:KubernetesIoTEdgeWorkingGroup} is a step in the right direction to funnel the industry effort and explore the possibilities of Kubernetes for the edge. Solutions which are tightly integrated into Kubernetes but do not run Kubernetes itself are going to have a hard time. Developers are going to expect the same experience and tools from their edge devices as from the cloud. I expect Kubernetes to take over the other projects and become the de-facto standard for edge and fog orchestration. In the authors opinion edge has to be promoted to a first class citizen in Kubernetes. The testing servers should support and test Kubernetes code on ARM. THe Kubernetes installer needs a new flag for less powerful devices, e.g. \textit{--edge=true}, which would automatically set the kubectl into powersave mode (e.g. reduce the housekeeping rate and more). It is time to look over the edge of the cloud and enable a common building ground as close to the IoT and mobile devices as possible.

This thesis led to code commits to the Istio as well as to the Kubeedge project. I would like to continue the work on envoy proxy and expand on the services implemented in this thesis. Additionally, I would like to explore first hand the possibilities of multi-cluster and multi-cloud architectures. It will be interesting to see how fast edge developers will embrace Kubernetes and create new tools for edge first and cloud second.
I will end this thesis with a quote from the CEO of the Eclipse Foundation summarizing the state of Kubernetes in 2019 on the edge:
\begin{displayquote}
\textit{\textbf{\Huge{``}}}
\textit{\large{
This is just getting started and this is a process — and not an event.\cite{ioFogMainBlog:online}
}}
\textit{\textbf{\Huge{''}}}
\\[1pt]
\raggedleft{{\rm --- Mike Milinkovich, CEO of Eclipse Foundation}}
\end{displayquote}
