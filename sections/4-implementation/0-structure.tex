\clearpage
\section{Implementation} \label{sec:implementation}
The implementation is based on one Kubernetes master in the cloud and and one worker on-premises, at the edge. The master runs on a virtual private server (VPS) from Contabo\cite{online:ContaboWebsite}. 
It has four cores, a minimum of 10 gigabyte of RAM, more then the official Kubernetes requirments\cite{kubernetesRequirementsInstall:online}, and runs Ubuntu 18.10. The worker is a raspberry pi (RPi) model 3b+ on bare metal as here performance is critical. It has one gigabyte of RAM and runs on ARMv7, a 32-bit architecture. It meets the Kubernetes system requirements for workers on paper, but it is generally not recommended by the Kubernetes developers to run Kubernetes for production purposes on the device. The RPi is one of the most popular single-board computers and often used in IoT project, I use it in this thesis as well.

I will divide the implementation in different stages for better understanding. In stage one I setup the system, installing the edge device and joining it to the master. In stage two, I will deploy an example application to the edge device controlled from the Kubernetes master. I will explain in more detail the build steps for the containers and how it is deployed. In stage three the IoT setup will be deployed. This includes a CoAP server on the edge and microcontroller sending data to that server. 


\subsection{Kubernetes Installation}
The Kubernetes master installation is well documented and did not pose any troubles. For completion of this thesis I will briefly mention the additional components installed on the master. Kubernetes has no default networking implementation, thus I installed Flannel\cite{coreosFlannel:online}, one of the simplest and most widely used networking fabrics specifically designed for Kubernetes. It runs a small binary called \textit{flanneld} on each node and provides networking between nodes and pods\footnote{The specifics of Flannels networking are outside this thesis scope, for more information see \url{https://blog.laputa.io/kubernetes-flannel-networking-6a1cb1f8ec7c}}. For administrative purposes I also installed the Kubernetes Dashboard.

The Raspberry Pi Foundation recently released an update to Raspbian (the RPi OS) called buster, which does dropped support of the aufs-dkms kernel modules, which are dependencies of kubeadm and containerd. The online community advised in forums to avoid the dependency but if possible switch back to the older version of Raspbian called stretch. Out of time concerns, I switched back to the old version. 

I switched the official Kubernetes kubelet with the kubelet binary from Ranchers K3s called Hyperkube. \Cref{fig:kubeBinaries} shows the size of the binary files of three most important components of Kubernetes, kubeadm, kubelet and kubectl.
\begin{figure}[h!]
    \centering
    \includegraphics[scale=0.5]{figures/kubeBinariesSize.png}
    \vspace*{-0.3cm}
    \mycaption[The Sizes of the Kubernetes Binaries.]{\\ Note, \textit{kubelet} is the K3s hyperkube binary and \textit{kubelet.old} is the official kubelet bianry.}
    \label{fig:kubeBinaries}
\end{figure}
In total the official binaries are 165 megabyte large and the kubelet (\textit{kubelet.old} file) takes almost half the space. Using the K3s build of kubelet (\textit{kubelet} file) brings the total amount to 138 megabyte saving almost 20\% of space in addition to its other benefits discussed in \cref{sec:Kubernetes} \nameref{sec:Kubernetes}. The full installation procedure can be found in the repository of this thesis for replication and validation.

Finally, the effects on the system resources are two folds, the native applications and the running containers. The Raspberry Pi has 926MB total memory and a 1.4 GHz quad-core CPU. Firstly, \cref{fig:kubernetesResourceConsumption} shows the resource usage of the native applications
\begin{figure}[h!]
    \centering
    \includegraphics[scale=0.5]{figures/kubernetesResourceConsumption.png}
    \vspace*{-0.3cm}
    \caption[The Resource Usage of native Kubernetes Components.]{Using the K3s hyperkube binary.}
    \label{fig:kubernetesResourceConsumption}
\end{figure}
The kubelet process uses 6.6\% of the CPU and 5.9\% of the memory. While the memory usage is consistent, the CPU usage varies a lot and can reach peaks of 16\%. Using the command
\begin{displayquote}
\textit{top | egrep --line-buffered '(kubelet|PID)' > /home/pi/kubeletHyper.txt }
\end{displayquote}
the cpu usage was tracked over 38 minutes with an average of 3.16 data points per second and a total of 7235 data points\footnote{The entire statistics can be found in the project files together with the source code.}. The average CPU usage in the time period was 10.3\%, the memory usage was constant.

Flanneld and kube-proxy both use less than 1\% CPU and 1.3\% and 1.6\% of memory, respectively. It is important to note that the container runtime is not included in that statistics. The resource usage of the Kubernetes relevant containers is shown in \cref{fig:kubernetesResourceConsumptionCut}\footnote{The figure is cut to fit the page. The full figure can be seen the appendix.}.
\begin{figure}[h!]
    \centering
    \includegraphics[scale=1.6]{figures/kubeContainerResourceUsageCut.png}
    \vspace*{-0.3cm}
    \caption[The Resource Usage of Kubernetes Containers.]{\\The command used is: \textit{docker stats | grep -E '(ID|kube|flannel)'}.}
    \label{fig:kubernetesResourceConsumptionCut}
\end{figure}
Flannel and kube-proxy consume 0.86\% and 1.96\% of the memory, respectively\footnote{flannel's limit is 50MB, hence the 16.02\% in the figure.}. The CPU usage of all containers is negligible at below 1\%. Adding up the numbers, after the installation Kubernetes uses 11.4\% of the CPU and 11.7\% of the total memory. This translates to 0.46GHz CPU usage, see \cref{eq:cpuTotal}, and 108.35MB memory usage, see \cref{eq:ramTotal}.
\begin{equation} \label{eq:cpuTotal}
    1.4GHz * 4 \; \textrm{(cores)} * 0.114 = 0.638GHz  \quad \textrm{(CPU usage)}
  \end{equation}
  \begin{equation} \label{eq:ramTotal}
    926.1MB * 0.117 = 108.354MB  \quad \textrm{(RAM usage)}
  \end{equation}
The memory consumption is especially concerning as swap needs to be disabled for Kubernetes to install. Swap is a partion in Linux used for paging. If disabled, memory can not be written to disk anymore which can lead to a memory overflow.
\subsection{Implementing a test Service} \label{sec:testService}
To test the Kubernetes setup I developed a simple go microservice called \textit{hello}. It listens on port 3000 and has three endpoints \textit{/hello/}, \textit{/hello/info} and \textit{/hello/path}. They don't implement any deeper logic but pose as an exemplary implementation for future services. I show exemplary Kubernetes configuration files and use this section to explain their meaning in more detail. For easier testability with cURL this microservice is based on json and not on protobufs.\\
\Cref{lst:dockerfileHello} shows the Dockerfile for building the microservice.
\lstset{
numbers=left, 
basicstyle=\footnotesize,
frame = single, 
language=Pascal, 
framexleftmargin=16pt,
% captionpos=b,
xleftmargin=2.3cm,
}
\begin{lstlisting}[linewidth=13cm, caption={Dockerfile for the Hello Application},label={lst:dockerfileHello}]
FROM golang:alpine as builder
EXPOSE 3000
COPY src .
RUN adduser -D -H -u 10001 scratchuser && \
    cd /go/main && \
    CGO_ENABLED=0 GOOS=linux GOARCH=arm GOARM=7 
    go build -a -installsuffix cgo -ldflags 
    '-extldflags "-static"' -o main .

FROM scratch
COPY --from=builder /etc/ssl/certs/ /etc/ssl/certs
COPY --from=builder /go/main/main /
COPY --from=builder /etc/passwd /etc/passwd
USER scratchuser
CMD ["./main"]
\end{lstlisting}
The code is cross-compiled in a builder container, which provides an isolated and replaceable build environment, this happens in line 1--8. Line 1 takes a fresh go alpine container with the latest go version. I then expose the port 3000 for the application to interact with its environment and copy the source code in the new container, line 2 and 3, respectively. Each new command produces a new temporary cached container, thus I the lines, 4--8, are all one \textit{RUN} command. It first adds a new user called \textit{scratchuser} without root privileges and the goes into the go main file directory and cross-compiles the application for ARMv7 (the last step is line 6--8). In lines 10--15, the final image is build. Line 10 initializes a new scratch container. This image is empty and thus uses a less memory and system resources. In lines 11--13 the relevant documents are copied in the new container. This includes the certificates from certificate authorities\footnote{They are important for encrypted connections.} (11), the application binary (12) and the user password (13). Line 14 switches to the scratchuser. This user has no root privileges and thus can never gain system control. Lastly, I specify the command run when a container based on the build image is run (15).\\
To see how important it is to use a scratch image as a base for memory purposes, consider \cref{fig:imageSizeComparison}. It shows just how much can be saved using this technique.
\begin{figure}[h!]
    \centering
    \includegraphics[scale=0.4]{figures/imageSizeComparison.png}
    \caption{The image sizes of the application.}
    \label{fig:imageSizeComparison}
\end{figure}
The go alpine image at the bottom is 350MB, the final image is a hefty 28MB bigger at 378MB (lines 1--8 in \cref{lst:dockerfileHello}). Copying the binary, the user files and the  certificates into a scratch image takes up only 6.76MB in total (lines 11--13). That is significantly smaller and with less overhead. When the container runs, only the binary is loaded and nothing else is running in the background. Contrast that to the alpine image where every time it is run, it starts a shell command line and includes an entire go build environment. The image is then tagged with a repository and tag name and pushed to the corresponding repository.\\
On the Kubernetes master I defined a service for the \textit{hello} application with the manifest shown in \cref{lst:serviceManifest}. It is an abstraction defining the guidelines how to access the pods inside a service, as the pods themselves are volatile\footnote{Each pod has a unique IP but Kubernetes does not guarantee for Pods and can reschedule pods at any moment}. It thus enables decoupling of the networking and the actual application from an outsiders perspective.
\lstset{
numbers=left, 
basicstyle=\footnotesize,
frame = single, 
language=Pascal, 
framexleftmargin=16pt,
% captionpos=b,
xleftmargin=2.3cm,
}
\begin{lstlisting}[linewidth=13cm, caption={The Service Manifest of the \textit{hello} Application.},label={lst:serviceManifest}]
apiVersion: v1
kind: Service
metadata:
  name: hello-service
  namespace: hello-namespace
spec:
  type: NodePort
  selector:
    app: hello
  ports:
    - name: http
      nodePort: 30001
      port: 3000
      targetPort: 3000
\end{lstlisting}
Line 5 tells the master that the service should be deployed in the namespace \textit{hello-namespace}. Line 6 specifies that each pod of this service should be accessible on the node it is scheduled on without going through the Kubernetes ingress resource. This is called nodeport and specified in line 12 to be \textit{30001}. Line 7 tells Kubernetes that the application corresponding to the service is called \textit{hello}. Finally, line 13 and 14 specify the ports the actual container expose.\\
For the actual state description of the application I use a \textit{Deployment} shown in \cref{lst:deploymentManifest}. A deployment specifies the desired state and the Deployment controller changes the actual state inside the cluster towards the desired state. 

\lstset{
numbers=left, 
basicstyle=\footnotesize,
frame = single, 
language=Pascal, 
framexleftmargin=16pt,
escapeinside=||,
xleftmargin=2.3cm,
}
\begin{lstlisting}[linewidth=13cm, caption={The Deployment Manifest of the \textit{hello} Application.},label={lst:deploymentManifest}]
apiVersion: apps/v1
kind: Deployment
metadata:
  name: hello-deployment
  namespace: hello-namespace |\Suppressnumber|
... |\Reactivatenumber{16}|
    spec:
      affinity:
          nodeAffinity:
            requiredDuringSchedulingIgnoredDuringExecution:
              nodeSelectorTerms:
                - matchExpressions:
                    - key: kubernetes.io/hostname
                      operator: In
                      values:
                        - pihome
        podAffinity:
          requiredDuringSchedulingIgnoredDuringExecution:
            - labelSelector:
                matchExpressions:
                  - key: env
                    operator: In
                    values:
                      - test
              topologyKey: "kubernetes.io/hostname"
      nodeSelector:
        pi: "hello" 
      containers:
        - name: hello
          image: jonas27/hello:v5arm |\Suppressnumber|
...
\end{lstlisting}

\comment{
Put this in Appendix

apiVersion: apps/v1
kind: Deployment
metadata:
  name: hello-deployment
  namespace: hello-namespace
spec:
  selector:
    matchLabels:
      app: hello
  replicas: 1
  template:
    metadata:
      labels:
        app: hello
        version: v5arm
    spec:
      affinity:
          nodeAffinity:
            requiredDuringSchedulingIgnoredDuringExecution:
              nodeSelectorTerms:
                - matchExpressions:
                    - key: kubernetes.io/hostname
                      operator: In
                      values:
                        - pihome
        podAffinity:
          requiredDuringSchedulingIgnoredDuringExecution:
            - labelSelector:
                matchExpressions:
                  - key: env
                    operator: In
                    values:
                      - test
              topologyKey: "kubernetes.io/hostname"
      nodeSelector:
        pi: "hello"
      containers:
        - name: hello
          image: jonas27/hello:v5arm
          imagePullPolicy: Always
          ports:
            - containerPort: 3000

}



Line 3 and 4 specify the name and namespace of the deployment. Lines 17--34 specify the pod affinities, which places a constrained on where a certain pod can be scheduled. Similarly, anti-affinities constrain a pod to where can not be scheduled\footnote{I deployed another pod to the node beforehand with the correct label}. Lines 18--25 specify a node affinity "it allows you to constrain which nodes your pod is eligible to be scheduled on, based on labels on the node"\cite{affinitiesKubernetes:online}.
Lines 26--34 specify the pod affinity. Pods of the deployment will only be scheduled on nodes containing a pod with the specified label. This enables orchestration based on other pods. Lines 35 and 36 specify the single node a deployment should be scheduled on. These three methods can be used together but have to chosen carefully. Finally, line 37--39 specify the container which should be deployed inside a pod (the port selection is hidden).\\
With this configuration it is possible to clearly specify the nodes a deployment should be scheduled on. I omitted how to accomplish it via namespace, which has many benefits to it as well. However, deployments, the resource type used in this section, are volatile deployments and should thus only be used for stateless application, like the \textit{hello} application. Stateful pods require the resource type \textit{StatefulSet} and pods supposed to run on every node require the resource type \textit{ReplicaSet}, see \cref{sec:statefulvsdeploymentvsBLAAAA} for more information.

\subsection{System Implementation}
The idea of the implementation is to create a light bulb connected to the cloud with pre-processing on the edge. Every time the light bulb is turned on or off the IoT device sends the status to the gateway. The gateway then creates a usage pattern and sends this statistics in intervals to the cloud for further processing and storage. This enables automated maintenance and more. The orchestration and management of the gateway and IoT device is done in the cloud where the light bulb can be controlled as well.

To reduce the overall system complexity only one device is deployed at each network layer. The architecture is shown in \cref{fig:actualSetup}. The IoT device, a esp32, is connected to a button and a light, the RPi makes up the edge layer and the VPS the cloud layer.
\begin{figure}[!ht]
    \centering
    \includegraphics[width=\textwidth]{figures/actualSetup.png}
    \caption{The Implemented Hardware Architecture}
    \label{fig:actualSetup}
\end{figure}
On the edge, all devices are directly connected either through wire or WiFi and are only dependent on each other to provide the normal functions. The expected round trip time (RTT) for a given signal should be rather low translating into an immediate response. In more extensive examples, the RPi could first check if pressing the button is even permissible and only then turn on the light/machine etc. In \cref{fig:actualSetup} the solid arrows stand for communication instantiated by the system and the dashed arrows communication instantiated by external changes. The systems internal communication normally goes up, from the IoT devices to the cloud, whereas system external communication goes the other way around. It is also expected that the system internal communication happens far more frequently than the system external one.

\Cref{fig:actualImplementationSetup} uses the same layout as \cref{fig:implementationSetup} but shows the actual system implementation. Comparing it to the desired system under \cref{sec:desiredSystem} shows that major and minor changes following the classification of the requirements had to be made as technology and time did not allow for a full implementation.

\begin{figure}[!ht]
    \centering
    \includegraphics[width=\textwidth]{figures/actualImplementationSetup.png}
    \caption{The Actual System Architecture}
    \label{fig:actualImplementationSetup}
\end{figure}

The cloud part is unchanged from the desired setup. The edge part had to be adopted due to technical issues. The Istio envoy proxy does not compile for the ARM architecture yet. Istio provides traffic shaping and routing, default mutual TLS encryption between services, an extensive ingress gateway and more. Kubernetes is already capable of some of these things but Istio is a far more potent tool. Its absence makes it impossible to reroute network traffic from the application to another destination and importantly the traffic between the edge and the cloud will not be encrypted. At the time of writing the Istio team is working on a patch for ARM devices. Using human produced certificates as an alternative is not a good idea, especially for small project as the certificates have to be kept up to date, should be rotated and mutual TLS is hard to implement.

Similarly, the CoAP librarby for esp32 does not support DTLS yet. The developers are aware of it and are working on fixing it. Protobufs are base 128 encode message which provides a shielding against to most rudimentary attacks, but it is not a security mechanism. Finally, due to time issues and prioritization the update service for the IoT device was not implemented.

Apart from the automatic updates for the edge device, the networking and encryption issues, all requirements labeled as SHOULD or MUST have been successfully implemented. By pressing a button a user can turn on or off a light. In the background, this data is sent from the esp32 to the RPi, which processes the data and sends a usage statistic of the light usage in defined time intervals to the server for further processing and storage. The system can also change the state of the light automatically from a centralized input field. Further, the administrator can update the specifications for all cluster internal resources (edge or cloud) and the system works towards implementing this specifications. It also means, in case of failure, e.g. an application crash on the edge, the system can reschedule this application. 

\begin{figure}[!ht]
    \centering
    \includegraphics[width=\textwidth]{figures/dashboardK8s.png}
    \caption{The Kubernetes Dashboard showing the Edge Application.}
    \label{fig:dashboarK8s}
\end{figure}

\Cref{fig:dashboarK8s} shows the Kubernetes dashboard of the running edge application. It shows that the deployments, pods and replica sets are all working as they should. More services for the same namespace would show up here as well. To see the status of the resources in other namespaces it is enough to change to that namespace, provided the permissions are granted.



\comment{
\bgroup\obeylines
Kubernetes in the cloud
Node running locally with HiveMQ on a raspberry pi
HiveMQ is running as container
Static vs non static ip (only static ip atm)
connected to esp32 devices
\egroup
}





\comment{
Kubeedge
OTA update to esp32
https://randomnerdtutorials.com/esp32-over-the-air-ota-programming/




}