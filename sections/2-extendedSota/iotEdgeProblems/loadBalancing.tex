Load balancing in a distributed environment is difficult as the ingress node needs a complete network topology at any given moment in time. This is one of the reasons why Kubernetes refreshes its node status so often. It also needs to synchronize this state across master nodes with a fast and distributed database called \textit{etcd}. External request go first to the master which needs to process and forward it to the right node. Load balancing is a perfect example why advantages and disadvantages of multi-cluster in contrast to single clusters have to be carefully weight. A full Kubernetes cluster on the edge comes with the advantage of having a full control plane on the edge and, thus, being able to load balance between nodes on the edge. 

But having a cluster at the edge comes with its downside. It needs a lot more management than a single cluster setup, consumes drastically more resources  and it needs a stable connection between the cluster nodes to work. But if resource consumption on the master is not an issue Kubernetes coupled with a load balancer provide more features and safeguarding than traditional (non orchestrated) load balancers do. Kubernetes always ensures that pods inside the cluster are healthy and reachable. If a pod goes down, Kubernetes will automatically schedule a new one. The internal load balancer can use this information and always route to an available pod. Other gateways like, Kong API Gateway or NGINX do not provide such features. Further, Kubernetes coupled with Istio makes it possible to do very fine grained traffic routing.

Finally, internal load balancing\footnote{Load balancing within one node.} is possible through normal API gateways and the deployment of multiple pods listening on different ports. For example, one NGINX container could function as load balancer for 5 containers on the same node. In Kubernetes this could be solved automatically with affinities. As soon as a node runs certain pods, a load balancer could automatically be side loaded.\\[5mm]