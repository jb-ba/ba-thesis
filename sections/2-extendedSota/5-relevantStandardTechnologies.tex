\subsection{Relevant Standardization}
Abstracting, managing and securing systems in the cloud is already mature with de-facto standard technologies. Containers, introduced in 2008 with LXC, abstract the operating system environment by assigning each container a new and separate user space. Programs inside a container have only access and can only see system resources provided to them\footnote{The kernel is the bare minimum in many container solutions.}. This means, they run (almost) as efficiently as any other application on the host without seeing or having access to system resources outside their provided scope.\\
In 2009, the first cluster management software, mesos, was presented, abstracting the orchestration from the underlying physical infrastructure mainly using containers. This means the deployment can be centralized and automated (deploy-time). In 2016 linkerd was introduced https://github.com/linkerd/linkerd/releases?after=0.3.1, the first service mesh. The service mesh abstracts the communication inside a cluster. Today, for each of these three aspects three opensource projects have emerged as defacto standards: ContainerD\footnote{ContainerD is the opensource container runtime originally developed by docker.} as container runtime, Kubernetes as container orchestration tool and Istio as service mesh. Together, these technologies make it easy to deploy and run a cloud environments consisting of many nodes.\\
Of course, IoT gateways have different requirements than servers. They are prone to unplanned downtimes, unstable Internet connections and use other protocols than the Internet standards IPv4 or IPv6. 