\subsubsection{Kubernetes} \label{kubernetesStandardization efforts}
Kubernetes is the de-facto standard for container orchestration and the main technology used in this thesis. It is a highly complex tool and offers a lot of configuration methods, thus understanding the fundamentals is highly important. A deep understanding of Kubernetes also helps in understanding the current shortcoming concerning the deployment on the edge. Kubernetes was designed from the ground up to be extensible, so most of the time, the question is not, if Kubernetes could do something, but rather if there is an existing implementation.\\
This section covers three important aspects of Kubernetes. First, single-cluster vs multi-cluster. Second, the core components of Kubernetes, mainly the kubelet, together with other resource configurations. Finally, the optimal deployment strategy, how to control and secure resources and how to select nodes.\\[0.5mm]
% \vspace{0.5mm} \ \\
\textbf{\textit{Single vs. Multi-Cluster}}\\
% \subsubsection{Single vs. Multi-Cluster} \label{sec:singleVsMultiCluster}
Single cluster Kubernetes provides many techniques to isolate and allocate resources. It is recommended for all but very big or multi-cloud deployments. It has limitations though which could be especially crucial on the edge. The official Kubernetes distribution has a multi-cluster management tool called Kubernetes Cluster Federation (KubeFed)\cite{kubernetesFederation97:online}. It provides another layer of fault isolation and because of multiple isolated control planes it enables low latency and scalability across regions. However, it does require additional management on top of Kubernetes already complicated configurations. It is thus important for bigger architectures with high system requirements, multi-cloud solutions and edge deployments.

Google recently revealed a new and much hyped product, Anthos\cite{TechnicalAnthosGoogle66:online}. It is a combination of many services, but most importantly it lets people easily manage multi-cluster deployments. Google describes it as follow: ``Anthos is a modern application management platform that provides a consistent development and operations experience for cloud and on-prem environments''\cite{TechnicalAnthosGoogle66:online}. Google provides an installer for on-prem devices but once installed Google itself manages the cluster including updates and troubleshooting. Not only is it a multi-cluster solution, but it also integrates the clusters with the Istio service mesh, providing unified communication and communication logs, and additional tools to ease the management. Anthos represents an industry wide trend for multi-cluster and multi-cloud solutions but it is closed source. Compared to Anthos, KubeFed lacks many of features for easier management and Istio, and is still in alpha status and thus not recommended for use in production environments.\\[0.5mm]



\comment{
These will not fade away and still be important. Finally, Docker's strong selling point of its cloud edge solution is its mirrored registry and neat features around it. It is possible though to setup a Docker Hub\footnote{The official Docker registry for images \url{https://hub.docker.com/}.} mirror for on-prem solutions. 


}


% \vspace{0.5mm} \ \\
\textbf{\textit{Core Components}}\\
% \subsubsection{Core Components}
Kubernetes is complex and resource intensive tool. However, much of the resource overhead stems from the control plane which only runs on the master nodes. These nodes are not in the scope of this thesis and thus not explained in more detail. Worker nodes on the other hand are orchestrated and provide their state to the master. This is a lot less resource intensive and there are a few tricks to reduce the resource load even further. For a worker node to be part of a cluster it needs to run the following components: the \textit{kubelet}, \textit{kube-proxy} and a supported container runtime. How Kubernetes works in practice can be seen in \cref{fig:nodeComponents}\footnote{The \textit{cAdvisor} component is a health checking tool and not required.}.
\begin{figure}[h!]
    \centering
    \includegraphics[scale=0.6]{figures/rancherK8sComponents.png}
    \caption{The Kubernetes architecture and node components\cite{nodeSetupKubernetes:online}}
    \label{fig:nodeComponents}
\end{figure}
The kubelet is the primary node agent and schedules and maintains containers running inside a pod based on the pods \textit{PodSpecs}. It gets the these specifications mainly from the APIServer, but other Kubernetes internal sources are possible, too. Containers created outside of the cluster (or multi cluster) are not managed by kubelet. Configuring the kubelet is easily possible via the kubectl and kubeadm tools which give the possibility of enhancing the kubelets performance for different circumstances, also the edge. Going through the possible configurations under \url{https://kubernetes.io/docs/reference/command-line-tools-reference/kubelet/}, the most important configurations for the edge is the \textit{--housekeeping-interval duration} which defaults to 10s\cite{rancherKubernetesComponents:online}. This means each 10 seconds kubelet performance a complete health check of all its components and sends it to the master. For normal nodes in the cloud, this is fine, but for light edge devices this seems like overkill. The kubelet from k3s (discussed in existing technologies) called hypervisor is compatible with Kubernetes and optimized for light edge devices.\\
For networking Kubernetes provides the kube-proxy which is a network proxy node agent ensuring that the Kubernetes networking services run on each node. It enables the Kubernetes service abstraction by ensuring the network rules on the host and carries out the connection forwarding. Kubernetes does not provide a standard implementation but requires the administrator to provide an implementation. The last component, the container runtime (discussed in the section before), ensures that containers can run as expected.\\[0.5mm]




% \vspace{0.5mm} \ \\
\textbf{\textit{Deployment Strategies}}\\
% \subsubsection{Deployment Strategies}
\comment{
deployment vs statefulset vs daemonset
clusterIP vs nodeport
Taints, nodeselector, affinities pod/node
namespaces
}
Kubernetes provides many API resources and the most important once for application deployments are discussed here.
The \textit{service} resource provides an abstraction between the network interface and the actual application. From an outsiders view, it is possible to call the Kubernetes ingress under an address subsection\footnote{E.g. In \url{https://example.com/hello} the path \textit{/hello} is the subsection} and get to a pod running somewhere inside the cluster without knowing its specific address. From the inside it enables routing between services based on the service name. Creating services is the usual deployment strategy for applications.

The service also defines how its pods are accessible. By default, pods are assigned a \textit{clusterIP} making it only accessible from inside the cluster. Outside requests have to go through the Kubernetes ingress and are then routed to a pod. This makes little sense on the edge, where latency is key and an Internet connection is not guaranteed. Instead, it is possible to define a \textit{nodePort} inside a service on a Kubernetes predefined port range. This port is then exposed by the host machine so that outside traffic can directly connect to the pod.

Kubernetes also offers different resources for core workloads, \textit{Deployments},  \textit{StatefulSets},  \textit{DaemonSets} and  \textit{ReplicationController}, which should not be used anymore\cite{CoreWorkloadKubernetes66:online}. These resources directly define what is running inside a pod. As with all Kubernetes API resources, they describe a desired state and the Kubernetes control plane works towards fulfilling this desired state. Deployments are mainly used for stateless services. StatefulSets are used for stateful services and DaemonSets deploy pods on each node. For the edge all workloads are important, but it is important to choose the correct workload for a desired result. Inside the workloads specifications it is also possible to define resources. Because of the resource limitations of edge devices auto-scaling is not possible and devices can get quickly overwhelmed by too many tasks. This also puts an emphasis on the prioritization of workloads. In case the hosts computing resources are not enough some workloads have to prioritized over others. 

Kubernetes offers powerful concepts to achieve the correct scheduling of pods on desired nodes. They are \textit{NodeSelector}, \textit{taints}, \textit{tolerations}, \textit{affinities} and \textit{anti-affinities}. A NodeSelector specifies which tags have to be present on a node for a pod to be scheduled on. Taints are added to nodes and specify that only pods with the matching toleration can run on the node. E.g. the master has the taint \textit{NoSchedule} which means, only pods with the matching toleration can be scheduled on a the master. Finally, affinities and anti-affinities, offer a way for pods to be scheduled (or not) on either pods or nodes with a specific tag. This gives operators the ability to add workloads only on nodes which are already running another pod, or with anit-affinity, where a pod is not present. These are very powerful administrative tools and need to be selected carefully.

Finally, \textit{namespaces} offer the ability to separate one physical cluster into mulitple virtual clusters. Most Kubernetes resources are saved inside namespaces which are especially important in a single-cluster setup with multiple user groups. Administrators can define namespaces and assign them resource limits as well as nodes via the \textit{PodNodeSelector}. This makes it possible to assign each logical edge deployment a virtual cluster and developers can only modify resources within that namespace or virtual cluster. It is also possible to define role-based access control (RBAC) to limit what a specific user or user group can modify inside a namespace. 

\\[0.5mm]
\textbf{\leftskip25mm\textit{Load Balancing}}\\
\subsubsection{Load Balancing}
Load balancing in a distributed environment is difficult as the ingress node needs a complete network topology at any given moment in time. This is one of the reasons why Kubernetes refreshes its node status so often. If a external request comes in at the master it needs to know where to forward the traffic. In the previous section \cref{sec:singleVsMultiCluster} \nameref{sec:singleVsMultiCluster} I discussed the advantages and disadvantages of multi-cluster in contrast to single clusters and load balancing is a perfect example why the decision has to be carefully weight. A full Kubernetes cluster on the edge comes with the advantage of having a full control plane on the edge and, thus, being able to do load balacing on the edge, with for example Istio Ingress Gateay. If the master nodes taint were to be removed it could even become part of the operational unit of the cluster.\\
But having a cluster at edge comes with its downside. It needs a lot more management than a single cluster setup. It consumes drastically more resources than having only worker nodes and it needs a stable connection between the cluster nodes to work. But if resource consumption on the master is not an issue Kubernetes coupled with a load balancer provide more features and safeguarding than most other load balancers do. Kubernetes always ensures that pods inside the cluster are healthy and reachable and if a pod goes down, Kubernetes will automatically spawn a new one. The internal load balancer can use this information and always route to an available pod. Other gateways like, Kong API Gateway, can not do this. Also with Istio it becomes possible to do very fine grained traffic routing, somthing I will discuss in the next section \cref{} \nameref{}.\\
Finally, internal load balancing\footnote{Load balancing withing one node.} is possible through normal API gateways and the deployment of multiple pods listening on different ports. For example, one NGINX container could function as load balancer for 5 docker containers in the background. In Kubernetes this could be solved with affinities. As soon as a node runs certain pods, a load balancer could be automotically side loaded.
\textbf{\leftskip25mm\textit{Traffic Control and Shaping}}\\
\subsubsection{Traffic Control and Shaping}
Traffic control and traffic shaping are very interesting edge topics. In Kubernetes each pod can be assigned incoming and outgoing bandwidth rates. But other tools extend this functionality and open up new possibilities. Istio, a service mesh mainly developed for Kubernetes, injects a sidecar proxy to fine tune the traffic of each container. It makes it possible to do rate limiting, control headers, reroute connection, change retries attempts, circuit breaking, mirroring and more while not changing the application code. Additionally, it is also able to encrypt traffic inside the mesh without the application knowing about it and gather all the telemetry data of the cluster. \\
With Kubernetes and tools extending its capabilities it is possible to do both traffic control and traffic shaping on a fine grained level. But, especially Istio, comes at the cost of additional overhead and the operator has to decide if the added functionality are worth the performance hit. Without Istio traffic control and shaping is only possible through the Kubernetes Ingress and thus in multi-cluster solutions.\\
Finally, Kubernetes is only meant to work with the Internet protocol stack and is only meant to operate within the clusters boundaries. So the actual communication with the IoT devices can not be controlled or shaped with Kubernetes. 

\comment{
Removed section: Security and Recommendations
IIoT and the smartphone revolution have accelerated the need for fog computing and the industry is set to develop a container based orchestration tool, tightly integrated with Kubernetes. 
IIoT has accelerated the need for fog computing and the industry is set to develop a container based orchestration tool, tightly integrated with Kubernetes. The Kubernetes IoT edge working group was set up specifically to find the optimal Kubernetes strategy for the edge. However, it is not clear, whether the solution will be Kubernetes itslef or a more specialized tool. In this thesis I will argue that Kubernetes is the way forward as it already posses many of the features that are need for the edge.\\ 
}