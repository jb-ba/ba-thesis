\subsubsection{Containers} \label{sec:containers}
Containers first surfaced in 2007 with the release of Linux Containers (LXC). They provide an abstraction and isolation layer to the software inside, hence the name containers. Basically, a container is packaged software that runs independent from the underlying system. They are created at runtime from container images, which are packaged, standalone executable software including everything the main application needs to run (code, runtime, system tools, libraries and settings)\cite{containerDefinition:online}. Containers offer
"unprecedented agility in developing and running applications in cloud environment especially when combined with a microservice-style architecture."\cite{microserviceContainers} Since their inception there has been considered development of containers, especially by Docker, a company developing the same named application which is often used synonymously for containers. Docker donated its main container runtime environment, called containerd, to the CNCF open source community and later, in late 2017, announced the first major release of containerd\footnote{See \url{https://blog.docker.com/2017/12/cncf-containerd-1-0-ga-announcement/} for more information on this topic}. Containerd is becoming rapidly the industry norm as the runtime was already widely used when it belonged to Docker.

Today, containers are the building blogs of the cloud and power almost all distributed applications. Craig McLuckie, the lead product manager cloud computing product at Google, said in a panel discussion during the Linux collaboration summit in Febraury 2015:
\begin{displayquote}
\textit{\textbf{\large{``}}}
\textit{This containers revolution is changing the basic act of software consumption. It’s redefining this much more lightweight, portable unit, or atom, that is much easier to manage... It’s a gateway to dynamic management and dynamic systems.}
\textit{\textbf{\large{''}}}
\end{displayquote}
It is important to note, that while the cloud saw the first transformation from containers, the edge is part of the same dynamic system and stands to profit enormously from adapting containers. They offer many advantages over conventional deployments but have their drawbacks. For the edge resource consumption, hardware access and security are especially important and explained in more detail.\footnote{To keep the thesis concise I will not explain how containers work in greater detail only when required for the understanding of the content. For more information see Docker's website  \url{https://www.docker.com/resources/what-container} and containerd \url{https://containerd.io/}}.\\[0.5mm]

% \vspace{0.5mm} \ \\
\textbf{\textit{Resource Consumption}}\\
% \subsubsection{Resource consumption}
Processing power consumption from containers is always greater than bare metal deployments. However, recent research has shown that in most cases this is almost negligible. In an updated performance assessments of containers the authors found that containers introduce almost no overhead over a native deployment\cite{felter2015updatedPerformanceContainers}. By design containers share little with the default namespace, apart from system resources, like the kernel. It is possible to give containers access to host directories, but this often breaks their design philosophy of being stateless and portable. The solution is to include the libraries, tools, certificates necessary to run the application in the container itself. This can potentially lead to a ballooning size of containers and with multiple containers running simultaneously, disk space and memory consumption can easily exceed the system resources of an edge devices. E.g. deploying the latest official openJDK container consumes 470MB and only includes the runtime dependency for the JVM. Source and byte code and dependencies of the application would all add extra disk space on top of this. Compiled language are at a huge advantage in this regard. They compile to machine code and include all code dependencies in one binary. Some compiled languages even offer the possibility of cross-compilation. This means source code can be compiled for a different architecture they are being compiled on. This allows servers running on the x86 or amd64 architecture to compile to native ARM machine code.

Additionally, containers allow and encourage multi-stage builds. This method of building containers allows to have one container to build the application and another one to run the application. The one used for building contains all dependencies and the compile environment, while the one used for running the application only contains the binary and run-time relevant files (e.g. certificates or the linux user). This eliminates any unused files and is often used in combination with the \textit{scratch} image. 

It is the base image for all other images and is basically an empty file structure consuming no disk space or memory\footnote{It also creates an isolated network and process namespace, but they are also very cheap.}. According to Docker the scratch image ``is most useful in the context of building [...] super minimal images''\cite{scratchImageDockerD65:online}. Another advantage using smaller images is the faster extracting time. Images are usually stored as archives and extracted on the host machine placing an additional strain on the system for each new deployment.\\[0.5mm]
\comment{
Combining these these techniques can result in far lower memory and RAM consumption, but also better security (discussed in the next section). In the implementation, \cref{sec:testService} \nameref{sec:testService}, I will show just how much memory can be saved using these techniques. 




In an updated performance assessments of containers the authors found that containers introduce almost no overhead over a native deployment\cite{felter2015updatedPerformanceContainers}. Hence, gateways running Linux can in most cases use containers without a problem. 
}

% \vspace{0.5mm} \ \\
\textbf{\textit{Hardware Access}}\\
% \subsubsection{Hardware Access}
Hardware access is very important on the edge, but not so much in the cloud. Servers rarely communicate via different protocols or use different peripherals. On the edge access to bluetooth or usb sticks is very important. Sharing host resources with a container is possible through the \textit{privileged} flag set at deploy time. In a nutshell, a container has its own filesystem, network and process tree and cannot access resources outside of this scope. Using the \textit{privileged} flag gives a container access to the host containers resources. Containerd offers many ways to fine tune how much privileges the container should get. It is for example possible to only expose the hosts network namespace and not the process namespace or filesystem. 

\comment{
This is done in the implementation \cref{} \nameref{} to reduce security risks. However, these resources are not controlled by Kubernetes and need to be managed by the operator.


}

% \vspace{0.5mm} \ \\
\textbf{\textit{Security}}\\
% \subsubsection{Security}
Security is one of the biggest challenges in IoT. In 2016 a botnet with over 600k infected devices, primarily edge and IoT devices, overwhelmed several high-profile targets like the DNS provider Dyn with massive distributed denial-of-service (DDoS) attacks. This caused a temporary outage of their DNS servers rendering many webpages, among others  Twitter, Spotify and Amazon, temporally inaccessible. \\
In a recent talk at the Container Security Conference 2019 Andy Martin, co-founder of Control Plane, noted ``The root of all evil is unnecessarily running processes or containers as root''\cite{RootlessContainerSecurityTalk0:online}. This user can read, write and run all files in all directories. In the beginning of 2018 a new vulnerability on Kubernetes was discovered giving containers with volume mounts the ability to access data outside the volumes sub-path. The root user would thus have unlimited access to the host resources. To combat this, the user should be changed to a non-root and non-sudo enabled user. Running a container as privileged as in the previous section explained has further security implications and should only be used if necessary. It breaks down the isolation layers and allows the container to access and modify the hosts resources. Additionally, not providing a shell inside containers reduces the attack surface even further. It is also not possible to expose new ports from inside a container.\\
Using a compiled language adds further security. Binaries are resistant to code injection as well as other modifications and reverse engineering. Even if someone got a copy of the image, they could not detect how it works or alter it. Combined with the image digest they could not even sideload another application an repackage it as this would change the image digest. \\
Containers are usually downloaded from registries and most registries allow for overriding of image versions. If malicious attackers gain access to the registry they could potentially overwrite the latests version for a specific version tag. This would mean, all downloads by version tag would then download the malicious version. However, each image also has an image digest, a cryptographic signature ensuring that the image in question is the actual image. In production an image should always be pulled based on the image digest, also called "Pinning-by-Digest". So instead of unsing the download link \url{jonas27/hello:v1.0.1} the administrator should specify the image digest like so \url{jonas27/hello@sha256:725355347bc2eae8f7c9ab9dd09b9ef2c884b3458c693978f5408785aef1fb72}. This simple technique not only greatly improves security but also guarantees that each container is derived from the same image and combats race conditions.\\


\comment{
To combat malicious attacks certain methods have been developed. I will only analyze the direct container security aspects as they are always applicable for containers. Other methods, like specifically developed OSs and securing the CI/CD pipeline are not discussed at this point.\\

In the implementation \cref{sec:testService} \nameref{sec:testService} I will use the \textit{hello} service for testing. It's current version, v5arm, is saved on the Docker image hub and can be pulled with the link \textit{jonas27/hello:v5arm}. 

% \footnote{In this thesis I will only show deployments of containers with tags as they are referenced in multiple places for clarity. With Helm Charts this is profoundly easier but outside the scope of this thesis.} 


In conclusion, containers can significantly increase the isolation if some basics rules are followed. Use compiled language if possible, use the scratch image, do not include or expose more than needed and most importantly, don't use the root user if not absolutely necessary. When system resources are exposed it is even more important to properly secure the container. 
}



\comment{
https://www.iotforall.com/containers-on-the-edge/

 They isolate the application runtime and the 

Containers and Cloud: From LXC to Docker to Kubernetes

%  try use if vspace does not work \\[0.5mm]
}
