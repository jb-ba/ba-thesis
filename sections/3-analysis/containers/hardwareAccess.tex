\subsubsection{Hardware Access}
Resource consumption and security are at least as important on the edge as in the cloud. The last container related topic, hardware access, is even more important on the edge. In the cloud rarely do devices need to communicate via different protocols or use different peripherals. For example, running a Bluetooth enabled container needs the access to bluetoothd on Linux and the hosts network. This is done with the \textit{privileged} flag. In a nutshell, a container has its own filesystem, network and process tree and cannot access resources outside of this scope. Using the \textit{privileged} flag gives a container access to the host containers resources. Containerd offers many ways to fine tune how much privileges the container should get. It is for example possible to only expose the hosts network namespace and not the process namespace or filesystem. This is done in the implementation \cref{} \nameref{} to reduce security risks. However, these resources are not controlled by Kubernetes and need to be managed by the operator.