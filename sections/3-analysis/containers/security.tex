\subsubsection{Security}
Security is one of the biggest challenges in IoT. In 2016 a botnet with over 600k infected devices, primarily edge and IoT devices, overwhelmed several high-profile targets like the DNS provider Dyn with massive distributed denial-of-service (DDoS) attacks. This caused a temporary outage of their DNS servers rendering many webpages, among others  Twitter, Spotify and Amazon, temporally inaccessible. To combat malicious attacks certain methods have been developed. I will only analyze the direct container security aspects as they are always applicable for containers. Other methods, like specifically developed OSs and securing the CI/CD pipeline are not discussed at this point.\\
Most registries allow for overriding of images. In the implementation \cref{sec:testService} \nameref{sec:testService} I will use the \textit{hello} service for testing. It's current version, v5arm, is saved on the Docker image hub and can be pulled with the link \textit{jonas27/hello:v5arm}. However, each image also has an image digest, a cryptographic signature ensuring that the image in question is the actual image. In production (not done in this thesis) an image should always be pulled based on the image digest, also called "Pinning-by-Digest". In the case of the test service the link would then be \textit{jonas27/hello:sha256:725355347bc2eae8f7c9ab9dd09b9ef2c884b3458c693978f5408785aef1fb72}. This simple technique not only improves security but also guarantees that each container is derived from the same image, but it also guarantees that every instance of the service is running exactly the same code and combats race conditions.\\
In a recent talk at the "Container Security Conference" Andy Martin, co-founder of control plane, noted ``The root of all evil is unnecessarily running processes or containers as root''\cite{RootlessContainerSecurityTalk0:online}. By default, all images use the root user as default user. However, the root user can read, write and run all files in all directories. In the beginning of 2018 a new vulnerability on Kubernetes was discovered giving containers with volume mounts the ability to access data outside the volumes sub-path. The root user would thus have unlimited access to the host resources. To combat this, the user can be changed to a newly created user. 

Scratch user
binary no reverse engineering
no modification possible.