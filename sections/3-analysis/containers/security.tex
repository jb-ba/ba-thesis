\subsubsection{Security}
Security is one of the biggest challenges in IoT. In 2016 a botnet with over 600k infected devices, primarily edge and IoT devices, overwhelmed several high-profile targets like the DNS provider Dyn with massive distributed denial-of-service (DDoS) attacks. This caused a temporary outage of their DNS servers rendering many webpages, among others  Twitter, Spotify and Amazon, temporally inaccessible. To combat malicious attacks certain methods have been developed. I will only analyze the direct container security aspects as they are always applicable for containers. Other methods, like specifically developed OSs and securing the CI/CD pipeline are not discussed at this point.\\
Most registries allow for overriding of images. In the implementation \cref{sec:testService} \nameref{sec:testService} I will use the \textit{hello} service for testing. It's current version, v5arm, is saved on the Docker image hub and can be pulled with the link \textit{jonas27/hello:v5arm}. However, each image also has an image digest, a cryptographic signature ensuring that the image in question is the actual image. In production\footnote{In this thesis I will only show deployments of containers with tags as they are referenced in multiple places for clarity. With Helm Charts this is profoundly easier but outside the scope of this thesis.} an image should always be pulled based on the image digest, also called "Pinning-by-Digest". In the case of the test service the link would then be \textit{jonas27/hello@sha256:725355347bc2eae8f7c9ab9dd09b9ef2c884b3458c693978f5408785aef1fb72}. This simple technique not only improves security but also guarantees that each container is derived from the same image. It also ensures that every instance of the service is running exactly the same code and combats race conditions.\\
In a recent talk at the Container Security Conference 2019 Andy Martin, co-founder of Control Plane, noted ``The root of all evil is unnecessarily running processes or containers as root''\cite{RootlessContainerSecurityTalk0:online}. By default, all images use the root user as default user. This user can read, write and run all files in all directories. In the beginning of 2018 a new vulnerability on Kubernetes was discovered giving containers with volume mounts the ability to access data outside the volumes sub-path. The root user would thus have unlimited access to the host resources. To combat this, the user should be changed to a non-root and non-sudo enabled user as shown in \cref{sec:testService} \nameref{sec:testService}. Running a container as privileged as in the previous section explained has further security implications and should only be used if necessary. It breaks down the isolation layers and allows the container to access and modify the entire system.\\
Using the \textit{scratch} image also reduces the attacking surface even when the host is compromised. They include no command line and thus make it next to impossible to modify their data. It is also not possible to expose new ports on a running container. Using a compiled language adds further security. The image only contains almost only the application binary which is resistant to code injection as well as other modifications and reverse engineering. Even if someone got a copy of the image, they could not alter it or detect how the application works. Combined with the image digest they could not even sideload another application an repackage it as the original as it would change the image digest. \\
In conclusion, containers can significantly increase the isolation if some basics rules are followed. Use compiled language if possible, use the scratch image, do not include or expose more than needed and most importantly, don't use the root user if not absolutely necessary. When system resources are exposed it is even more important to properly secure the container. 