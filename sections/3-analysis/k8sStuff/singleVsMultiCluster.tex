\subsubsection{Single vs. Multi-Cluster}
Google recently revealed a new and much hyped product, Anthos\cite{TechnicalAnthosGoogle66:online}. Its a combination of many services, but most importantly it lets people easily manage hybrid clouds. Google describes it as follow: ``Anthos is a modern application management platform that provides a consistent development and operations experience for cloud and on-prem environments''\cite{TechnicalAnthosGoogle66:online}. Google provides an installer for on-prem but onece installed Google itself manages the cluster including updates and troubleshooting. Not only is it a multi-cluster solution, but it also integrates the clusters with Istio service mesh and other neat features. The industry trend is clear multi-cluster, multi-cloud solutions are the way forward. However, they still require a lot of management on top of the normal Kubernetes and thus mostly important for bigger companies. Open source Kubernetes has a multi-cluster management tool called Kubernetes Cluster Federation (KubeFed)\cite{kubernetesFederation97:online}. It is not as integrated as Anthos, however, and still in alpha as of time of writing. Multi-cluster solutions provide another layer of isolation. For example, a master in an on-prem deployment could never schedule pods in the cloud. However, analyzing these multi-cluster tools and testing them is outside the scope of this thesis. "Simple" Kubernetes already provides many techniques to isolate and allocate resources. These will not fade away and still be important. Finally, Docker's strong selling point of its cloud edge solution is its mirrored registry and neat features around it. It is possible though to setup a Docker Hub\footnote{The official Docker registry for images \url{https://hub.docker.com/}.} mirror for on-prem solutions. 
