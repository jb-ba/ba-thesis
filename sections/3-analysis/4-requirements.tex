\subsection{Requirements}
The requirements are based on the analysis subsections before and provide a guideline for the implementation in the next section. I will first collect the requirments in functional requirements. The functional requirements table \cref{table} will already contain a proposed technical solution and a reasoning for the selected technology. I will not create non-functional requirements, as this thesis aim is to provide an exploratory example of using cloud orchestration tools at the edge. Non-functional requirements "indirectly related to the overall success of the system"\cite{aauFunctionalRequirements} and are usually confirmed by intense testing.\\
After providing the functional requirements I will order them with the MoSCoW method\cite{clegg1994caseMoSCoWMethod} and, finally, I will conclude the analysis section with a short summary.

\comment{
First list and then clissify (MoSCoW

An example of a functional requirement would be:
A system must send an email whenever a certain condition is met (e.g. an order is placed, a customer signs up, etc).

Docker:
From Scratch
no privileges, only if really important and only what is needed
no root, only if necessary
Compiled language
Only necessary files in container

Kubernetes:
Single cluster as small
Docker Hub mirror
Use a edge ready kubelet
Use namespace to separate into virtual clusters

IoTEdgeProblems
Assign priorities for traffic
Use Istio to secure communication and for traffic shaping
Use CoAP to explore kubernetes possiblities.
Use protobufs.






}
