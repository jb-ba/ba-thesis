\subsection{Requirements Specification}
The requirements are based on insights from existing solutions and the use cases and provide a guideline for the implementation in the next section. I will only create functional requirements as non-functional requirements "indirectly related to the overall success of the system"\cite{aauFunctionalRequirements} and are usually confirmed by intense testing. As this thesis aim is to provide an exploratory example of using Kubernetes at the edge specific constraints are not important. 

The functional requirements are listed and ordered via the MoSCoW method in \cref{tab:functionalRequirements}. The main purpose in this thesis is to have a working prototype showcasing Kubernetes on the edge. The MUST requirements are the once deemed necessary to fulfill this and have a "M" (for MUST) in the "Class" field of the table. The SHOULD requirements are abbreviated by "S" are features which do not make or brake the system, but for an industry wide standard they should be present. This mainly includes security and networking aspects. The COULD requirements are mainly features which add extra business functionality and make the system easier to use. The WOULD requirements are features which are either not directly related to the product or out of scope.
% Please add the following required packages to your document preamble:
% \usepackage[normalem]{ulem}
% \useunder{\uline}{\ul}{}
% \begin{table}[]
\clearpage
\setlength\LTleft{-2.5cm}
\begin{longtable}{l p{3.5cm} p{0.8cm} p{12.5cm} }
\multicolumn{4}{l}{Functional Requirements}       
                                                                                                                                                                                                                              \\
ID                      & Name                                    & Class  & Description                                                                                                                                                                                                                         \\ \hline




101                     & Isolated applications                   & M      & Each application has to function on its own. It has to handle database errors and communication errors in a non fatal way.                                                                                                          \\
102                     & Isolation to host computer              & M      & The application has to be as isolated from the host environment as possible.                                                                                                                                                        \\
103                     & Minimal dependencies                    & M      & The application has to have minimal runtime dependencies. If the application is built on the edge device the compile time dependencies have to be minimal as well.                                                                  \\
104                     & Cross compilation                       & M      & The edge applications has to be able to be cross-compiled.                                                                                                                                                                          \\
105                     & Small images                            & M      & The footprint of the application has to be small as possible                                                                                                                                                                        \\
106                     & Use different users                     & M      & Each application should be run by different users. If one userspace is compromised it does not effect the other applications.                                                                                                       \\
107                     & Use non root user                       & M      & The application should not run with root priviledges except when really needed.                                                                                                                                                     \\
108                     & Isolate the application context         & M      & The application process has to be isolated. Only if the application must access host resources should they be available to the application.                                                                                         \\
109                     & Secured application code                & M      & The application source code has to be secure in case of a databreach on the node. A malicious attacker should not be able to reconstruct the source code even when he gets access to it.                                            \\
110                     & Trusted applications                    & W      & The edge application would be verified before the deployment.                                                                                                                                                                       \\
111                     & Centralized orchestration               & M      & The operator should only need to tell a central configuration tool in the cloud how the system should looks like. The cloud has to then orchestrate the edge automatically.                                                         \\
112                     & Automatic application deployment        & M      & Based on a new configuration the edge device has to be able to automatically get a copy of the new application shut down the old application and deploy the new one.                                                                \\
113                     & Isolate edge from cloud                 & C      & The cloud and the edge could work independently so without a data connection. This is in case the edge is powerful enough to run its own control plane and the setup is big enough to justify the overhead of running multicluster. \\
114                     & Isolated cloud environments             & S      & Developers should only have access to the resources they need.                                                                                                                                                                      \\
115                     & Resource quotas                         & M      & The operators must be able to set resource quotas on applications and groups of applications.                                                                                                                                       \\
116                     & ARM compatible edge                     & M      & The orchestration tool on the edge has to be able to run on (slower) ARM powered devices.                                                                                                                                           \\
117                     & Node fault tolerance                    & M      & The state of each node must be synchronized in the system so that in case of an error the last operable state can be recovered.                                                                                                     \\
118                     & Decentralized edge storage              & W      & When possible each edge device would synchronization important data with other edge devices in a distributed database.                                                                                                              \\
119                     & Traffic prioritization                  & S      & System critical information should be prioritized in case of congestion.                                                                                                                                                            \\
120                     & Whitelist connections                   & S      & Only cluster internal communication should be allowed to access applications on the pods.                                                                                                                                           \\
121                     & Application independent traffic shaping & S      & The traffic for each application should be contralable without changing the application code.                                                                                                                                       \\
122                     & Secure traffic within cluster           & S      & The communication between the nodes inside the cluster should be secure. This includes the edge to cloud communication.                                                                                                             \\
123                     & Internet compatible IoT protocol        & M      & The IoT protocol must to be compatible with the Internet protocol. This enables addressing of IoT devices through the Internet in case of a static IP.                                                                              \\
124                     & Secure traffic with IoT devices         & S      & The communication between the IoT gateway/edge device and the IoT device needs to be secure.                                                                                                                                        \\
125                     & Efficient message compression           & M      & The messages between the edge device and the IoT device must be efficiently compressed to save resource and enhance speed.                                                                                                          \\
126                     & Energy efficient transmission           & M      & The transmission between the edge and the IoT device must be energy efficient.                                                                                                                                                      \\
127                     & Automatic updates                       & S & The edge device should be able to update the IoT device. The update process is intiated in the cloud but the operator does not have to do local configurations. The system updates alone once initiated.                            \\
128                     & OTA updates                             & C      & The IoT device updates could support OTA updates.                                                                                                                                                                                   \\
129                     & Verified updates                        & W  & The IoT device updates would be verified before installing them.                                                                                                                                                                   




\end{longtable}
\label{tab:functionalRequirements}
% \end{table}
\clearpage

\comment{
First list and then clissify (MoSCoW

An example of a functional requirement would be:
A system must send an email whenever a certain condition is met (e.g. an order is placed, a customer signs up, etc).

Docker:
From Scratch
no privileges, only if really important and only what is needed
no root, only if necessary
Compiled language
Only necessary files in container

Kubernetes:
Single cluster as small
Docker Hub mirror
Use a edge ready kubelet
Use namespace to separate into virtual clusters

IoTEdgeProblems
Assign priorities for traffic
Use Istio to secure communication and for traffic shaping
Use CoAP to explore kubernetes possiblities.
Use protobufs.






}
