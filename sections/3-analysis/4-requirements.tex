\subsection{Requirements Specification}
The requirements are based on insights from existing solutions and the use cases and provide a guideline for the implementation in the next section. I will only create functional requirements as non-functional requirements "indirectly related to the overall success of the system"\cite{aauFunctionalRequirements} and are usually confirmed by intense testing. As this thesis aim is to provide an exploratory example of using Kubernetes at the edge specific constraints are not important. \\
The functional requirements are listed and ordered via the MoSCoW method in \cref{tab:functionalRequirements}.

\comment{
First list and then clissify (MoSCoW

An example of a functional requirement would be:
A system must send an email whenever a certain condition is met (e.g. an order is placed, a customer signs up, etc).

Docker:
From Scratch
no privileges, only if really important and only what is needed
no root, only if necessary
Compiled language
Only necessary files in container

Kubernetes:
Single cluster as small
Docker Hub mirror
Use a edge ready kubelet
Use namespace to separate into virtual clusters

IoTEdgeProblems
Assign priorities for traffic
Use Istio to secure communication and for traffic shaping
Use CoAP to explore kubernetes possiblities.
Use protobufs.






}
