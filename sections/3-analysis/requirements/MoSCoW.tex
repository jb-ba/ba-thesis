


https://projekter.aau.dk/projekter/files/281557079/Report_final.pdf
Preface
The project is made on 10. semester Network and Distributed Systems and has been
carried out in the spring of 2018 at Aalborg University(AAU). This project addresses
the design of an access control protocol for Industry 4.0 utilizing blockchain.
This project has been supervised by Jens Myrup Pedersen (associated professor at
AAU). The authors would like to thank AAU Smart Production for the resources
provided for the project.
The reader is expected to have fundamental knowledge corresponding to an 10thsemester student at Network and Distributed Systems(NDS), Aalborg University.
However, prior knowledge about blockchain and access control is not assumed.
Sources are referenced [n] according to IEEE citation reference standard, where n is
a number represented in the bibliography, which is placed at the end of the report.
The appendices are placed after the bibliography.
This project is made as a part of the joint European collaboration Improving
Employability through Internationalization and Collaboration (EPIC) [1]. The
collaboration is an Erasmus+ project [2]. The goal of EPIC is to make students from
different universities in Europe collaborate on joint projects. In spring 2018 an EPIC
project, Security in Internet of Things (IoT), has been carried out.
The project Security in Internet of Things is a combination of two master thesis:
“Access Control for Industry 4.0 – Initial Trust with Blockchain” by Jacob Kjersgaard
and Martin Eriksen from Aalborg University (AAU) and “Development of Conceptual
Trust Handling Framework for Industry 4.0 Wireless Networks” by Marina
Harlamova from Riga Technical University (RTU). The focus of the EPIC project is
security in industrial IoT which is a key concept in Industry 4.0. The project was
established with a joint analysis of a formulated problem. The analysis is then used
as a theoretical basis for further work in individual theses, therefor has chapters 1
Introduction and 2 The Case of Industry 4.0, Marina Harlamova as a third author.
This report contains the common analysis, as well as the work made by Jacob
Kjersgaard and Martin Eriksen from Aalborg University (AAU).
In total three different writings, the master thesis from RTU, the EPIC report and
this master thesis, are produced. The master theses are delivered to the respected
university and the joint report is handed to EPIC.