\subsection{Containers} \label{containers}
Containers first surfaced in 2007 with the release of Linux Containers (LXC). They provide an abstraction and isolation layer to the software they are supposed to run. Basically, a container is packaged software that runs independent from the underlying system. They are created at runtime from container images, which are packaged, standalone executable software including everything the main application needs to run (code, runtime, system tools, libraries and settings)\cite{containerDefinition:online}. Thus, containers offer
"unprecedented agility in developing and running applications in cloud environment especially when combined with a microservice-style architecture."\cite{microserviceContainers} Since 2007, there has been considered development of containers, especially by Docker, a company developing the same named application which is often used synonymously for containers. Docker than donated its main container runtime environment, called containerd, to the open source community and later, in late 2017, announced the first major release of containerd. https://blog.docker.com/2017/12/cncf-containerd-1-0-ga-announcement/ This was significant news, as the docker software was already widely used and it marked the start of an independent open source standard for containers. \\
Today, containers are the building blogs of the cloud and power almost all distributed applications. Craig McLuckie, the lead product manager cloud computing product at Google, said in a panel discussion during the Linux collaboration summit in Febraury 2015:
\begin{displayquote}
\textit{\textbf{\large{``}}}
\textit{This containers revolution is changing the basic act of software consumption. It’s redefining this much more lightweight, portable unit, or atom, that is much easier to manage... It’s a gateway to dynamic management and dynamic systems.}
\textit{\textbf{\large{''}}}
\end{displayquote}
It is important to note, that while the cloud saw the first transformation from containers the edge is part of the same dynamic system and also needs dynamic management.\\ 
The rest of the section will discuss the advantages of using containers focusing on the edge component. I will show the advantages and disadvantages of containers in terms of memory management, portability and security, and ways to combat the disadvantages. To keep the thesis concise I will not explain how containers in greater detail\footnote{For more information on this topic see Docker's website  \url{https://www.docker.com/resources/what-container} and containerd \url{https://containerd.io/}} .

\subsubsection{Resource consumption}
Memory consumption from containers is always greater than bare metal deployments. By design containers share little with the default namespace, apart from system resources, like the kernel. It is possible to give containers access to host directories, but this often breaks their design philosophy of being stateless and portable. The solution is to include the libraries, tools, certificates necessary to run the application in the container itself. It is not hard to image with multiple containers memory and RAM consumption can easily exceed the system resources of edge devices, e.g. the Raspberry Pi’s. For example, deploying the latest official openJDK container consumes 470MB. Source and byte code and dependencies add on top of this. Compiled language are at a huge advantage in this regard. Go is one such example and used in this thesis. It can be cross-compiled to machine code for most CPU architecture compatible with Linux and results in one application file without any dependencies. Multi-stage builds allow to have one container to build the application and only copy relevant files into the final image. This eliminates any unused files and is often used in combination with the \textit{scratch} image. This is the base image for all other containers and is basically an empty file structure consuming no memory. According to Docker it ``is most useful in the context of building [...] super minimal images''\cite{scratchImageDockerD65:online}.\\
Combining these these techniques can result in far lower memory and RAM consumption, but also better security (discussed in the next section). In the implementation, \cref{sec:testService} \nameref{sec:testService}
hard topic, as this is not needed in the cloud
\subsubsection{Security}
Security is one of the biggest challenges in IoT. In 2016 a botnet with over 600k infected devices, primarily edge and IoT devices, overwhelmed several high-profile targets like the DNS provider Dyn with massive distributed denial-of-service (DDoS) attacks. This caused a temporary outage of their DNS servers rendering many webpages, among others  Twitter, Spotify and Amazon, temporally inaccessible. To combat malicious attacks certain methods have been developed. I will only analyze the direct container security aspects as they are always applicable for containers. Other methods, like specifically developed OSs and securing the CI/CD pipeline are not discussed at this point.\\
Most registries allow for overriding of images. In the implementation \cref{sec:testService} \nameref{sec:testService} I will use the \textit{hello} service for testing. It's current version, v5arm, is saved on the Docker image hub and can be pulled with the link \textit{jonas27/hello:v5arm}. However, each image also has an image digest, a cryptographic signature ensuring that the image in question is the actual image. In production (not done in this thesis) an image should always be pulled based on the image digest, also called "Pinning-by-Digest". In the case of the test service the link would then be \textit{jonas27/hello:sha256:725355347bc2eae8f7c9ab9dd09b9ef2c884b3458c693978f5408785aef1fb72}. This simple technique not only improves security but also guarantees that each container is derived from the same image, but it also guarantees that every instance of the service is running exactly the same code and combats race conditions.\\
In a recent talk at the "Container Security Conference" Andy Martin, co-founder of control plane, noted ``The root of all evil is unnecessarily running processes or containers as root''\cite{RootlessContainerSecurityTalk0:online}. By default, all images use the root user as default user. However, the root user can read, write and run all files in all directories. In the beginning of 2018 a new vulnerability on Kubernetes was discovered giving containers with volume mounts the ability to access data outside the volumes sub-path. The root user would thus have unlimited access to the host resources. To combat this, the user can be changed to a newly created user. 

Scratch user
binary no reverse engineering
no modification possible.



\comment{
https://www.iotforall.com/containers-on-the-edge/

 They isolate the application runtime and the 

Containers and Cloud: From LXC to Docker to Kubernetes

}
