\subsection{Kubernetes}
Kubernetes is the de-facto standard for container orchestration and the main aspect of this thesis. In this section I look at the fundamentals of Kubernetes and highlight the best strategies to run it on edge devices. I will also look what is currently missing to make it a truly edge ready. With that being said, Kubernetes has so many features and tools that discussing them all is impossible. The same is true for its ecosystem which is constantly evolving and adding even more layers to the mix. I will try to keep this section short and concise, but it will come at the cost of complexity.\\
IIoT has accelerated the need for fog computing and the industry is set to develop a container based orchestration tool, tightly integrated with Kubernetes. The Kubernetes IoT edge working group was set up specifically to find the optimal Kubernetes strategy for the edge. However, it is not clear, whether the solution will be Kubernetes itslef or a more specialized tool. In this thesis I will argue that Kubernetes is the way forward as it already posses many of the features that are need for the edge.\\ 
I will start with addressing the elephant in the room, single-cluster vs multi-cluster. I will then analyze the core components of Kubernetes, mainly the kubelet, together with other resource configurations. Afterwards, I will focus on the optimal deployment strategy, how to control resources on the edge and node selection. Finally, I will look at the security aspect of Kubernetes and conclude with recommendations for the implementation.

\subsubsection{Single Vs Multi-Cluster}
Google recently revealed a new and much hyped product, Anthos\cite{TechnicalAnthosGoogle66:online}. Its a combination of many services, but most importantly it lets people easily manage hybrid clouds. Google describes it as follow: "Anthos is a modern application management platform that provides a consistent development and operations experience for cloud and on-prem environments"\cite{TechnicalAnthosGoogle66:online}. Google provides an installer for on-prem but onece installed Google itself manages the cluster including updates and troubleshooting. Not only is it a multi-cluster solution, but it also integrates the clusters with Istio service mesh and other neat features. Open source Kubernetes only has the multi-cluster management solution called Kubernetes Cluster Federation.


