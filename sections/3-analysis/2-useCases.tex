\subsection{Use Cases}
Kubernetes can be installed on most devices running Linux including many single-board computers like the Raspberry Pi. Using Kubernetes comes with the benefit of full orchestration from the cloud. That means, once a desired state is provided to the Kubernetes master it will work towards achieving this desired state. If the system leaves the desired state, Kubernetes will actively try to restore it. 

The main downside of this is the processing overhead. Containers, the main deployment method of applications in Kubernetes have been shown to have little overhead over native applications, see \cref{sec:containers}. Thus, if an IoT gateway has sufficient resources to run Kubernetes and if the system does not have real time requirements, then Kubernetes can be an excellent choice. Following are two use cases for Kubernetes.


\subsubsection{Use Case 1: Smart Stores}
Smart stores have many cameras and sensors monitoring the consumers and items. This data needs to be constantly analyzed and converted into meta data. This meta data needs to be constantly available to all nodes inside the cluster.

The IoT gateways used in the store are heterogeneous, some are based on ARM and others on x86\_64. They also differ in processing power, network connection, the number of applications they run and the IoT devices they are connected to. Some of the IoT devices run on battery and have very limited processing power. There are system relevant applications which need to be always running. Others are for telemterics and non business related purposes and are not system critical.

The store also maintains some legacy apps which should be integrated into the cluster. For these application it is important that the traffic can be controlled without actually changing the application code.

As the store changes its offerings and layout the gateways are used to deploy a wide variety of  applications. Developers adding features to an application also deploy the new application, but are not able to change other resources. To ease the administration the store sets the configuration in the cloud and lets the system do the update. 

All applications and nodes have to be constantly working and one crash or malicious attack should not jeoperdize the entire system.

\subsubsection{Use Case 2: Ticket Terminal}
A cinema chain has a ticket terminals where the user can create an account and set his credit card details and other user data. The terminals are placed in each cinema and upload the user data to the cloud for centralized storage. Thus, a user in cinema A can also automatically check in in cinema B. The terminals also have a card reader for easier check in. Every year the card reader manufacturer produces an update for the devices. These updates are deployed on the terminals via the central configuration point.

Because of GDPR purposes all communication between card reader and the terminal and the cloud and the terminals is encrypted. The applications are also only able to communicate with the cloud.
The cinema company also made sure that the terminal and application are secure and that only trusted applications run on the terminal.

All terminals work independently from each other but run the same application code. Each month the design of the application is changed to represent new movies. The developers build a new image and push it to a registry. They then tell the system to deploy the new version.

\comment{
Since edge devices can also produce terabytes of data, taking the analytics closer to the source of the data on the edge can be more cost-effective by analyzing data near the source and only sending small batches of condensed information back to the centralized systems.
}
