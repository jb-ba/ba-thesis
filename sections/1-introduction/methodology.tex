\subsection{Methodology}
The methodology describes the theoretical background for the methods applied in this thesis and ensures consistency across related work in the field.
The section Extended State of the Art is meant to provide the reader with an objective and complete overview of the research area. Following, in the Analysis section use cases are used to gather requirements. These are then prioritized according to the importance and feasibility for the desired system. Finally, in the Implementation, the system based on these requirements is implemented following a software development method and the priorities of the requirements.\\[5mm]
\textbf{\leftskip25mm\textit{Existing Solutions}}\\
The existing solutions section gives an objective presentations of already existing solutions in the field and sheds a light on how widely adopted systems or new once solve particular problems. It is not about the product but rather what the product tries to achieve and how.\\[5mm]
\textbf{\leftskip25mm\textit{Use Cases}}\\
Use cases are a way to find requirements\cite{UseCase94Fowler:online}. They can appear in the form of UML diagrams or written text. In this thesis only the written text version is used as the UML diagrams "[...]are of little value" and "the key value of use cases lies in the text" according to Martin Fowler\cite{UseCase94Fowler:online}. As the developed system is only a prototype I will use the "casual template for low-ceremony projects" from the book "Writing Effective Use Cases" from Alistair Cockburn\cite{cockburn2000writingUseCases}.\\[5mm]
\textbf{\leftskip25mm\textit{Functional Requirements}}\\
Functional Requirements state what services a system should provide, how it reacts to particular inputs, and how it should behave under particular circumstances\cite{sommerville2011software}. They are derived from the use cases and the state of the art and are the foundation for the implementation.\\[5mm]
\textbf{\leftskip25mm\textit{MoSCoW}}\\
The MoSCoW method is a way to structure the requirements according to the needs of the stakeholders involved\cite{sommerville2011software}. It is an acronym for "Must, Should, Could, Would". It helps to plan time and resources in a project so that the requirements with a higher ranking are finished first.\\[5mm]
\textbf{\leftskip25mm\textit{Agile Software Development Method}}\\
The software development method is a way to structure the development of software. An agile method emphasize development cycles to increase the transparency and flexibility during the software development. It decreases the risk of failure by constantly reevaluating the product. The agile manifest developed by industry experts\cite{beck2001manifestoAgile} has four main values: Individuals and Interactions over processes and tools; Working Software over comprehensive documentation; Customer Collaboration over contract negotiation; Responding to Change over following a plan. These are the foundation for a working agile software development method.
