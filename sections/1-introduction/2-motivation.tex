\subsection{Motivation}
Edge computing, and especially fog computing, are emerging as a new paradigm on how to structure the computational resources of a system. It enables technology to operate very close to the user or thing and places no additional burden on the core network or servers. According to Dejan Bosanac, a senior software engineer at Red Hat in the field of cloud messaging and IoT platforms, these devices have three main advantages over the cloud due to their proximity to the devices and end users: 
\begin{displayquote}
{\textbf{``Low latency, availability and locality''}}\cite{IntroducingDejanBosanac:KubernetesIoTEdgeWorkingGroup}
\end{displayquote} 
Edge computing is set to have profound changes on mobile computing and Industrial IoT (IIoT) and has been described as "enabler for the Industrial Internet of Things"\cite{steiner2016fogenablerIIoT} in the academic literature.

However, there is no industry wide standard for structuring or deploying edge resources. No software has yet emerged and taken the industry by storm and became the de-facto standard. Kubernetes has done so in the cloud and I am going to argue Kubernetes will be the technology to transform the edge from an isolated into an active part in the data processing pipeline. Importantly, Kubernetes not only has the advantage of fog computing, i.e. cloud aware edge device or enabling offline edge clusters, but also includes central and standardized monitoring and networking. As most systems will be connected to the Internet one way or another in the future, fog computing stands to be the primary use case, but it is not required.

For edge computing security, hardware restrictions, isolation, fault tolerance and more are immensely important issues and Kubernetes already has the tools and features to solve many of these problems. Importantly, Kubernetes allows for remote controlling and monitoring of systems and works actively to ensure the desired state of the cluster.

\comment{


In this thesis I will thus explore how Kubernetes can facilitate fog computing and what challenges still have to be solved.
}