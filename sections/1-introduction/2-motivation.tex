\subsection{Motivation}
Edge computing, and especially fog computing, are emerging as a new paradigm on how to structure a systems architecture. It enables technology to operate very close to the user or thing while not placing an extra burden on the core network. According to Dejan Bosanac, a senior software engineer at Red Hat in the field of cloud messaging and IoT platforms, due to their proximity to the sensors and the end user, these devices have three main advantages over the cloud: 
\begin{displayquote}
{\textbf{``Low latency, availability and locality''}}\cite{IntroducingDejanBosanac:KubernetesIoTEdgeWorkingGroup} - Dejan Bosanac.
\end{displayquote} 
Edge computing is set to have profound changes on IIoT and in the academic literature has been described as "enabler for the Industrial Internet of Things"\cite{steiner2016fogenablerIIoT}.\\
But there is no unified way of doing edge computing. No software has yet emerged and taken the industry by storm and became the de-facto standard. Kubernetes has done so in the cloud world and I am going to argue Kubernetes will be the technology to transform the edge from an isolated into an active part in the data processing pipeline. Importantly, Kubernetes not only has the advantage of fog computing, i.e. cloud aware edge device, but also enables offline clusters of only edge devices. As most systems will be connected to the Internet one way or another in the future, fog computing stands to be the primary use case, but it is not required.\\
For edge computing security, hardware restrictions, isolation, fault tolerance and more are immensely important issues and Kubernetes already has the tools and features to solve many of these problems. Importantly, Kubernetes allows for remote controlling and monitoring of the system and always tries to bring the system to its desired state.\\

\comment{


In this thesis I will thus explore how Kubernetes can facilitate fog computing and what challenges still have to be solved.
}