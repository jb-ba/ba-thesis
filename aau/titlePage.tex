\pdfbookmark[0]{English title page}{label:titlepage_en}
\aautitlepage{%
  \englishprojectinfo{
    Orchestration on the Edge: Exploring Kubernetes for Edge Computing
  }{%
    Bachelor of Science Thesis
  }{%
    Spring 2019
  }{%
    ITCOM 6.10
  }{%
    %list of group members
    Jonas Burster
  }{%
    %list of supervisors
    Assoc. Prof. Henning Olesen
  }{%
    1 % number of printed copies
  }{%
    \today % date of completion
  }%
}{%department and address
  \textbf{Technical Faculty of IT and Design}\\
  Aalborg University\\
  \href{https://www.en.tech.aau.dk/}{https://www.en.tech.aau.dk/}
}{% the abstract
  \small{ By 2025 the vast amount of data will be produced by devices belonging to the Internet of Things (IoT). These devices are integrated into everyday objects and are used to enable new technologies, like autonomous cars and factories. If these devices were to send their data directly to the cloud, it could topple not just individual servers but the entire system.
  
  Edge computing is the proposed solution to this problem. Traditionally, the gateways the IoT devices connect to just forwarded their information to a centralized system in the cloud. In edge computing the gateway forms an active part in the data processing pipeline. New system architectures emerged tightly coupling the edge resources to resources in the cloud. But outsourcing business logic on the edge, comes with its challenges. These devices, notoriously known for bad updates, now need to be orchestrated and managed.
  
  The purpose of this thesis is to show how Kubernetes, the de-facto standard for cloud orchestration, can help with orchestration and management on the edge. It will explain important resources and concepts in connection to Kubernetes and provide an exemplary implementation of a Kubernetes orchestrate cloud connected edge system.}
}