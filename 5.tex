\section{Argument which type of Knowledge Management and Enterprise Systems that may best suit the needs of the case organization (10\%)}
Agillic needs to cope with a wide variety of requirements for its Knowledge Management (KM) systems. The main forms of knowledge are tacit knowledge, project specific knowledge, organizational knowledge, company knowledge and external knowledge. \cite{jashapara2004knowledge} presents 5 different KM systems: Document management systems, decision support systems, group support systems, workflow support systems and customer relationship management systems.\\
For simplicity and considering the size of Agillic, there should be one big platform for providing extensive internal documentation. Jira \citep{jira} is a very popular documentation tool for agile project management and a good fit for Agillic. It can be used by the whole company without any modifications and supports different access right schemes for spaces (i.e. a sub-directory). To coordinate the workflow a simple email or chat tool, e.g. Slack \citep{slack} is enough.\\
The teams need to tackle various problems: how to get the right information to the right person at the right time, how to enhance the communication, knowledge sharing, cooperation, coordination, social encounters within groups and how to manage associated workflows \citep{jashapara2004knowledge}. Having a good documentation platform is enough for most organizations, however developers need an extra platform 
for version-control  (e.g. git) to share code and handle associated workflows via pull/merge requests. It is often a problem to keep documents updated between the version-control system and the documentation platform. Jira  has a plugin for gitlab and github, web based git managers, which do exactly that. They mirror the content between the two systems, so that the developer does not need to worry about updating both systems. Tacit knowledge is often not documented because the effort of doing so is greater than its benefits. So having an easy to use solution for taking notes inside the version-control system would be of great benefit. \\ 
Agillic should also introduce guidelines for good documentation including best practices on how to write and store documents. Similarly to opinionated software in the previous chapter, documentation guidelines reduce the documentation overhead. It eases the burden of the writer to find a good style and make it easier for other employees to find and read the documents. A good example for this is the morphological box in \cref{fig:kees} by \cite{kees2015characteristics} where the field for Agillics version-control system are highlighted. 
It represents a unified approach on how to classify enterprise systems on their maturity, target group, technology, dissemination and contracting. This would pave the way for new enterprise or KM systems to be introduced when the company outgrows the current one two system solution, while still maintaining a clear structure for all stakeholders.